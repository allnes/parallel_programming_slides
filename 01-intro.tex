\documentclass{beamer}

% Theme choice
\usetheme{Madrid}

% Optional packages
\usepackage{graphicx} % For including images
\usepackage{amsmath}  % For math symbols and formulas
\usepackage{hyperref} % For hyperlinks
\usepackage{listings}
\usepackage{xcolor}
\usepackage[T1]{fontenc}

\lstdefinestyle{CStyle}{
  language=C,                    % Set the language to C
  basicstyle=\ttfamily\footnotesize\linespread{0.9}\tiny, % Set font style and size
  keywordstyle=\color{blue},      % Color of keywords
  commentstyle=\color{gray},      % Color of comments
  stringstyle=\color{red},        % Color of strings
  showstringspaces=false,         % Do not mark spaces in strings
  breaklines=true,                % Enable line breaks at appropriate places
  breakatwhitespace=false,        % Break lines at any character, not just whitespace
  numbers=left,                   % Show line numbers on the left
  numberstyle=\tiny\color{gray},  % Style for line numbers
  tabsize=4,                      % Set tab width
  keepspaces=true,                % Keep indentation spaces
  frame=single,                   % Add a border around the code
  aboveskip=0pt,                  % Reduce space above the code block
  belowskip=0pt,                   % Reduce space below the code block
  xleftmargin=7.5pt,                      % Add left padding (approx. 2.8mm or 10px)
  xrightmargin=15pt,                      % Add left padding (approx. 2.8mm or 10px)
}

% Title, author, date, and institute (optional)
\title[Parallel Programming. Introduction]{Parallel Programming course. Introduction}
\author{Obolenskiy Arseniy, Nesterov Alexander}
\institute{Nizhny Novgorod State University}

\date{\today} % or \date{Month Day, Year}

% Redefine the footline to display both the short title and the university name
\setbeamertemplate{footline}{
  \leavevmode%
  \hbox{%
    \begin{beamercolorbox}[wd=.45\paperwidth,ht=2.5ex,dp=1ex,leftskip=1em,center]{author in head/foot}%
        \usebeamerfont{author in head/foot}\insertshortinstitute % Displays the university name
    \end{beamercolorbox}%
    \begin{beamercolorbox}[wd=.45\paperwidth,ht=2.5ex,dp=1ex,leftskip=1em,center]{author in head/foot}%
      \usebeamerfont{author in head/foot}\insertshorttitle % Displays the short title
    \end{beamercolorbox}%
    \begin{beamercolorbox}[wd=.1\paperwidth,ht=2.5ex,dp=1ex,rightskip=1em,center]{author in head/foot}%
      \usebeamerfont{author in head/foot}\insertframenumber{} / \inserttotalframenumber
    \end{beamercolorbox}}%
  \vskip0pt%
}

\begin{document}

% Title slide
\begin{frame}
    \titlepage
\end{frame}

% Table of Contents (optional)
\begin{frame}{Contents}
    \tableofcontents
\end{frame}

% Section
\section{Introduction to MPI}

% "Hello, World" in MPI
\begin{frame}[fragile]{"Hello, World" in MPI}

  \lstset{style=CStyle, caption=Basic application written using MPI}
  \begin{lstlisting}
#include <mpi.h>

#include <iostream>

int main(int argc, char** argv) {
  MPI_Init(&argc, &argv);

  int world_size;
  MPI_Comm_size(MPI_COMM_WORLD, &world_size);

  int world_rank;
  MPI_Comm_rank(MPI_COMM_WORLD, &world_rank);

  char processor_name[MPI_MAX_PROCESSOR_NAME];
  int len_chars;
  MPI_Get_processor_name(processor_name, &len_chars);

  MPI_Barrier(MPI_COMM_WORLD);
  std::cout << "Processor = " << processor_name << std::endl;
  std::cout << "Rank = " << world_rank << std::endl;
  std::cout << "Number of processors = " << world_size << std::endl;

  MPI_Finalize();
  return 0;
}
  \end{lstlisting}

\end{frame}

\begin{frame}[fragile]{Performance measurement in MPI: \texttt{MPI\_Wtime()}}
  \texttt{double MPI\_Wtime(void)}

  \begin{itemize}
    \item \texttt{MPI\_Wtime()} is a function provided by the MPI standard to measure the wall-clock time (in seconds) since some arbitrary point in the past.
    \item This function is often used for performance analysis in parallel programs to measure the execution time of sections of code.
    \item It returns a double precision floating-point number representing the current time. The returned time is in seconds.
    \item \texttt{MPI\_Wtime()} is local to the process and does not guarantee synchronization between processes, meaning each process may have a different starting time reference.
  \end{itemize}

  Usage example:
  \lstset{style=CStyle}
  \begin{lstlisting}
double start = MPI_Wtime();
// Code to time
double end = MPI_Wtime();
double elapsed = end - start;
  \end{lstlisting}

  Documentation reference: \texttt{\href{https://www.mpich.org/static/docs/v3.2/www3/MPI_Wtime.html}{https://www.mpich.org/static/docs/v3.2/www3/MPI\_Wtime.html}}
\end{frame}


% Thank You Slide
\begin{frame}
    \centering
    \Huge{Thank You!}
\end{frame}

% Optional references slide
\begin{frame}{References}
\end{frame}

\end{document}
